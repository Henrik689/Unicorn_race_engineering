\section{AVR programmering ved at bruge windows}
\subsection{Software krævet}
Til brænding med AVRISP mk2 i Windows kræves følgende software:
\begin{enumerate}
	\item[-]{Avr-dude:				http://sourceforge.net/projects/winavr/}
	\item[-]{AVRISP mkII libusb:	http://mightyohm.com/blog/2010/09/avrisp-mkii-libusb-drivers-for-windows-7-vista-x64/}
\end{enumerate}

For at installere driversne til AVRIPS mk2 ordentligt skal man i enhedshåndteringen i kontrolpanelet, manuelt ind og opdatere driversne til enheden, og led den frem til destinations mappen for AVRISP mkII libusb.

\subsection{Brænding af software}
Til brænding af software skal man i Windows være i ``tvunget'' admin rettigheder.
\newline
Dette gøres ved at man i cmd.exe skriver kommandoen:
\begin{lstlisting}
runas /noprofile /user:<brugernavn> cmd\newline
\end{lstlisting}
Dette åbner en nyt vindue som brændingen skal foregå i.\\
når man er i den rigtige mappesti til den node man ønsker at brænde software til, som f.eks kunne se sådanne ud:\\
\url{c:\\users\\<brugernavn>\\documents\\github\\Unicorn_race_engineering\\g5\\car_software\\node1} \\
når du er i den mappe som du ønsker at brænde kode for har du kommandoer der kan bruges:
\begin{lstlisting}
make
\end{lstlisting}
Bruges til kun at compile koden så man hurtigt kan se om der er fejl i det nye kode.
\begin{lstlisting}
make install
\end{lstlisting}
Brænder koden til chippen.
